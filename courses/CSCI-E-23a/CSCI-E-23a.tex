\section{CSCI-E-23a - Introduction to Game Development}

This course picks up where Harvard College’s CS50 leaves off, focusing on the development of 2D and 3D interactive games. Students explore the design 
of such childhood games as Super Mario Bros., Legend of Zelda, and Portal in a quest to understand how video games themselves are implemented. Via lectures and hands-on projects, the course explores principles of 2D and 3D graphics, animation, sound, and collision detection using frameworks like Unity and LÖVE 2D, as well as languages like Lua and C\#. By class’s end, students will have programmed several of their own games and gained a thorough understanding of the basics of game design and development.
\\\\
\textbf{Course Site:} \url{https://cs50.github.io/games/}
\\\\
This is Harvard's followup course to CS50 for Games Development. It's not yet available on edX but the lectures and assignments are available online so I'm following along. Languages: lua and C\#(Unity).
\newpage

\subsection{Assignment 0 - Pong}

\begin{featurebox}
\textbf{Highlights}
\begin{itemize}
\item Lua
\item LÖVE2D
\item Basic OOP (Object-Oriented Programming)
\item Drawing Shapes
\item Drawing Text
\item DeltaTime and Velocity
\item Game State
\item Box Collision (Hitboxes)
\item Sound Effects (with bfxr)
\end{itemize}
\end{featurebox}

\subsubsection{Lecture}


\begin{table}[h]
\caption{Important Links}
\label{pong-important-links}
\begin{tabular}{ll}
\hline
Purpose           & Link                                     \\ \hline
Installing love2d & \url{https://love2d.org\#download}             \\ 
Running love2d    & \url{https://love2d.org/wiki/Getting\_Started} \\ 
Demo code         & \url{https://github.com/games50/pong}          \\ 
Game Programming Patterns (online book) & \url{https://gameprogrammingpatterns.com}  \\
bfxr (create audio effects) & \url{https://www.bfxr.net/} \\
\hline
\end{tabular}
\end{table}


\paragraph{Important Concepts}

\begin{definition}
\textbf{Game Loop}
\\
The intention of the Game Loop pattern is to decouple the progression of game time from user input and processor speed. It processes input, updates the game, and then renders anything that has changed in the game's state. The important takeaway is that while it processes input, it doesn't wait for it. It is responsible for running the game at a consistent speed, regardless of where it's running (old pc, android device, etc).
\end{definition}

\begin{definition}
\textbf{2D Coordinate System}
\\
Typical cooridinate system, but with the origin $(x_0, y_0)$ at the top left of the system. Positive - to the right and down. Negative - to the left and up.
\end{definition}

\begin{definition}
\textbf{Class}
\\
A blueprint for encapsulating methods and attributes which allow you to instantiate objects which follow that blueprint. Each object created has it's own data. It is convention to capitalize class names. The ":init" method is used as a constructor. "self" is used in a class definition to reference the current object's instance variables.
\\
The advantage of Object Oriented programming with regards to games development is that it allows us to defer update and render logic to each object, keeping our main code much shorter and more maintainable.
\end{definition}

\begin{definition}
\textbf{AABB Collision Detection}
\\
Relies on colliding entities to not be rotated (have "axis-aligned bounding boxes" and enables the use of a simple formula to detect collisions between them.
\begin{lstlisting}[language={[5.0]Lua}]
if a.x is not > b.x + b.width and
   a.x + a.width is not < b.x and
   a.y is not > b.y + b.height and
   a.y + a.height is not < b.y:
   collision = true
else
   collision = false
\end{lstlisting}

For more info see: \url{https://developer.mozilla.org/en-US/docs/Games/Techniques/2D_collision_detection}
\end{definition}

\begin{definition}
\textbf{State Machine}
Monitors the current state of the game, what states are possible, and what transitions take place to effect a change from one state to another.
\\
Note: later look up Pushdown Automation \url{https://en.wikipedia.org/wiki/Pushdown_automaton} which employs a stack of states so that you can not only know the current state, but also prior states.
\end{definition}
\newpage
\begin{table}[h]
\caption{Important Functions - Love2D}
\label{pong-important-functions-love}
\begin{tabular}{p{6cm}p{7.5cm}}
\hline
Function           & Description                                     \\ \hline
love.audio.newSource(path, [type]) & creates an audio object which can be played at any point in the program. Type can be "stream" or "static" - static is held in memory, stream is read from disk. \\
love.draw() & renders any changes to the screen \\
love.event.quit() & terminates the game \\
love.graphics.clear(r, g, b, a) & fills the entire screen with the desired color \\
love.graphics.newFont(path, size) & loads a font into memory \\
love.graphics.printf(text, x, y, [width], [align]) & prints text to the screen, can align text left, right or center \\
love.graphics.rectangle(mode, x, y, width, height) & draws a rectangle to the screen, mode: fill or line \\
love.graphics.setColor(r, g, b, a) & sets the currently active color for drawing shapes or text \\
love.graphics.setDefaultFilter(min, mag) & texture scaling filter - bilinear(blurry), nearest(retro) \\
love.graphics.setFont(font) & sets the currently active font \\
love.keypressed(key) & callback, executes when key pressed \\
love.keyboard.isDown(key) & returns true or false depending on whether the specific key is currently being pressed \\
love.load() & initializes game state            \\ 
love.resize(width, height) & Called by Love2d when we resize the application. Any code that needs to be run if the window is resized (like push:resize() goes here) \\
love.timer.getFPS() & returns the current FPS of the application \\
love.update(dt) & called each frame / dt is elapsed time since last frame \\
love.window.setMode(width, height, params) & window init, typically use push library instead \\
love.window.setTitle(title) & sets the title of the application window \\
\hline
\end{tabular}
\end{table}

\newpage

\begin{table}[h]
\caption{Important Functions - Other}
\label{pong-important-functions-other}
\begin{tabular}{p{6cm}p{7.5cm}}
\hline
Function           & Description                                     \\ \hline
push.setupScreen(VW, VH, WW, WH, { params }) & initialize virtual resolution, params table:  fullscreen, resizable, vsync \\ 
push.apply("start") & begin rendering at virtual resolution \\
push.apply("end") & end rendering at virtual resolution \\
math.randomseed(num) & seeds the random number generator used by math.random \\
math.random(min, max) & returns a random number between min and max \\
math.min(num1, num2) & returns the lesser of num1 and num 2 \\
math.max(num1, num2) & returns the greater of num1 and num 2 \\
os.time() &  returns the the time since Jan 1, 1970 in seconds \\
 &   \\ \hline
\end{tabular}
\end{table}
\newpage
\subsubsection{Assignment}

The assignment for Pong was pretty simple - add an AI player to control one or both of the paddles. The way I tackled this was to add a few more attributes to the paddles to make them more generic (like what controls a paddle uses, and whether it's controlled manually or by automated), then handle movement based on the state of those variables. Snippets:

\begin{lstlisting}[language={[5.0]Lua}]
    -- player 1
    if player1.autoplay == true then
        autoPlay(player1)
    else
        processPlayerInput(player1)
    end

    -- player 2
    if player2.autoplay == true then
        autoPlay(player2)
    else
        processPlayerInput(player2)
    end
\end{lstlisting}

\begin{lstlisting}[language={[5.0]Lua}]
--[[
    A function that will cause the passed player's Paddle 
    to be be moved according to player input
]]
function processPlayerInput(player)
    if love.keyboard.isDown(player.upkey) then
        player.dy = -PADDLE_SPEED
    elseif love.keyboard.isDown(player.downkey) then
        player.dy = PADDLE_SPEED
    else
        player.dy = 0
    end
end   

--[[
    A function that will cause the control of the passed player's
    paddle to be automated
]]
function autoPlay(player)
    if ball.y > player.y + player.height / 4 then
        player.dy = PADDLE_SPEED
    end
    if ball.y < player.y  - player.height / 4 then
        player.dy = -PADDLE_SPEED
    end
    if ball.y > player.y and ball.y < player.y + player.height then
        player.dy = 0
    end 
end   
\end{lstlisting}
\newpage
\subsection{Assignment 1 - Flappy Bird}

\begin{featurebox}
\textbf{Highlights}
\begin{itemize}
\item Images (Sprites)
\item Infinite Scrolling
\item “Games are Illusions”
\item Procedural Generation
\item State Machines
\item Music
\item Mouse Input
\end{itemize}
\end{featurebox}

\subsubsection{Lecture}

\subsubsection{Assignment}

\begin{comment}
\newpage
\subsection{Assignment 2 - Breakout}

\begin{featurebox}
\textbf{Highlights}
\begin{itemize}
\item Sprite Sheets
\item Procedural Layouts
\item Levels
\item Player Health (Loss Conditions)
\item Particle Systems
\item Collision Detection Revisited
\item Persistent Save Data
\end{itemize}
\end{featurebox}

\subsubsection{Lecture}

\subsubsection{Assignment}

\newpage
\subsection{Assignment 3 - Match 3}

\begin{featurebox}
\textbf{Highlights}
\begin{itemize}
\item Anonymous Functions
\item Tweening
\item Timers
\item Solving Matches
\item Procedural Grids
\item Sprite Art and Palettes
\end{itemize}
\end{featurebox}

\subsubsection{Lecture}

\subsubsection{Assignment}

\newpage
\subsection{Assignment 4 - Super Mario Bros.}

\begin{featurebox}
\textbf{Highlights}
\begin{itemize}
\item Tile Maps
\item 2D Animation
\item Procedural Level Generation
\item Platformer Physics
\item Intro to AI
\item Powerups
\end{itemize}
\end{featurebox}

\subsubsection{Lecture}

\subsubsection{Assignment}

\newpage
\subsection{Assignment 5 - Legend of Zelda}

\begin{featurebox}
\textbf{Highlights}
\begin{itemize}
\item Top-Down Perspective
\item Infinite Dungeon Generation
\item Events
\item Hurtboxes
\item Screen Scrolling
\item Data-Driven Design
\end{itemize}
\end{featurebox}

\subsubsection{Lecture}

\subsubsection{Assignment}

\newpage
\subsection{Assignment 6 - Angry Birds}

\begin{featurebox}
\textbf{Highlights}
\begin{itemize}
\item Box2D
\item Mouse Input
\end{itemize}
\end{featurebox}

\subsubsection{Lecture}

\subsubsection{Assignment}

\newpage
\subsection{Assignment 7 - Pokemon}

\begin{featurebox}
\textbf{Highlights}
\begin{itemize}
\item StateStacks
\item GUIs
\item Turn-Based Systems
\item RPG Mechanics
\end{itemize}
\end{featurebox}

\subsubsection{Lecture}

\subsubsection{Assignment}

\newpage
\subsection{Assignment 8 - 3D Helicopter Game}

\begin{featurebox}
\textbf{Highlights}
\begin{itemize}
\item Unity 3D and Editor
\item C\#
\item Blender
\item Components
\item Colliders and Triggers
\item Prefabs and Spawning
\item Texture Scrolling
\item Audio
\end{itemize}
\end{featurebox}

\subsubsection{Lecture}

\subsubsection{Assignment}

\newpage
\subsection{Assignment 9 - Dreadhalls}

\begin{featurebox}
\textbf{Highlights}
\begin{itemize}
\item 3D Texturing
\item First-Person Camera
\item VR with Unity
\item Procedural Maze Generation
\end{itemize}
\end{featurebox}

\subsubsection{Lecture}

\subsubsection{Assignment}

\newpage
\subsection{Assignment 10 - Portal}

\begin{featurebox}
\textbf{Highlights}
\begin{itemize}
\item Raycasting
\item Texture Decals
\item Render to Texture
\item Projectiles
\item 3D Physics
\end{itemize}
\end{featurebox}

\subsubsection{Lecture}

\subsubsection{Assignment}

\newpage
\subsection{Project 0 - Space Invaders}
\newpage
\subsection{Project 1}
\newpage
\subsection{Project 2}
\newpage
\subsection{Final Project}
\end{comment}